% Options for packages loaded elsewhere
\PassOptionsToPackage{unicode}{hyperref}
\PassOptionsToPackage{hyphens}{url}
%
\documentclass[
]{article}
\usepackage{lmodern}
\usepackage{amssymb,amsmath}
\usepackage{ifxetex,ifluatex}
\ifnum 0\ifxetex 1\fi\ifluatex 1\fi=0 % if pdftex
  \usepackage[T1]{fontenc}
  \usepackage[utf8]{inputenc}
  \usepackage{textcomp} % provide euro and other symbols
\else % if luatex or xetex
  \usepackage{unicode-math}
  \defaultfontfeatures{Scale=MatchLowercase}
  \defaultfontfeatures[\rmfamily]{Ligatures=TeX,Scale=1}
\fi
% Use upquote if available, for straight quotes in verbatim environments
\IfFileExists{upquote.sty}{\usepackage{upquote}}{}
\IfFileExists{microtype.sty}{% use microtype if available
  \usepackage[]{microtype}
  \UseMicrotypeSet[protrusion]{basicmath} % disable protrusion for tt fonts
}{}
\makeatletter
\@ifundefined{KOMAClassName}{% if non-KOMA class
  \IfFileExists{parskip.sty}{%
    \usepackage{parskip}
  }{% else
    \setlength{\parindent}{0pt}
    \setlength{\parskip}{6pt plus 2pt minus 1pt}}
}{% if KOMA class
  \KOMAoptions{parskip=half}}
\makeatother
\usepackage{xcolor}
\IfFileExists{xurl.sty}{\usepackage{xurl}}{} % add URL line breaks if available
\IfFileExists{bookmark.sty}{\usepackage{bookmark}}{\usepackage{hyperref}}
\hypersetup{
  pdftitle={A Broad Introduction to Bayesian Statistics},
  pdfauthor={Matthew Gonnerman},
  hidelinks,
  pdfcreator={LaTeX via pandoc}}
\urlstyle{same} % disable monospaced font for URLs
\usepackage[margin=1in]{geometry}
\usepackage{graphicx,grffile}
\makeatletter
\def\maxwidth{\ifdim\Gin@nat@width>\linewidth\linewidth\else\Gin@nat@width\fi}
\def\maxheight{\ifdim\Gin@nat@height>\textheight\textheight\else\Gin@nat@height\fi}
\makeatother
% Scale images if necessary, so that they will not overflow the page
% margins by default, and it is still possible to overwrite the defaults
% using explicit options in \includegraphics[width, height, ...]{}
\setkeys{Gin}{width=\maxwidth,height=\maxheight,keepaspectratio}
% Set default figure placement to htbp
\makeatletter
\def\fps@figure{htbp}
\makeatother
\setlength{\emergencystretch}{3em} % prevent overfull lines
\providecommand{\tightlist}{%
  \setlength{\itemsep}{0pt}\setlength{\parskip}{0pt}}
\setcounter{secnumdepth}{-\maxdimen} % remove section numbering

\title{A Broad Introduction to Bayesian Statistics}
\author{Matthew Gonnerman}
\date{1/16/2022}

\begin{document}
\maketitle

\emph{A Note to Readers: This module is a work in progress}

\hypertarget{outline}{%
\subsubsection{Outline:}\label{outline}}

\protect\hyperlink{objectives}{Objectives}\\
\protect\hyperlink{known-and-unknown-information}{Known and Unknown
Information}\\
\protect\hyperlink{a-frequentist-approach}{A Frequentist Approach}\\
\protect\hyperlink{a-bayesian-approach}{A Bayesian Approach}

\hypertarget{objectives}{%
\subsubsection{Objectives}\label{objectives}}

This module is intended as a starting point for learning about Bayesian
statistics, specifically aimed at graduate students studying ecology who
are unfamiliar or uncomfortable with statistics in general. It should
not be considered a comprehensive description but rather as a means of
entry that hopefully will give you confidence to further explore these
concepts. I have stripped away much of the details about methods and
derivation of equations to make this more of a summary of how different
statistical philosophies approach a problem. To that end, I wrote this
module with the following objectives in mind\ldots{}\\

\begin{enumerate}
\def\labelenumi{\arabic{enumi})}
\tightlist
\item
  To provide a broad, introductory description of Bayesian philosophy.\\
\item
  To breakdown the mechanisms by which Bayesian statistics functions as
  well as provide a brief summary of one of its most commonly used
  tools, Markov chain Monte Carlo simulations.\\
\item
  To differentiate a Bayesian approach from the more commonly taught
  Frequentist approach.\\
\item
  \emph{To force myself to better understand these concepts and to learn
  RMarkdown in the process (This one is just for me, but at least now
  you know why someone would do this in the first place).}
\end{enumerate}

\hypertarget{known-and-unknown-information}{%
\subsubsection{Known and Unknown
Information}\label{known-and-unknown-information}}

If you are reading this module, I assume you are either being forced to
or you are a graduate student with a need or desire to better understand
your options when conducting analyses. While we all are working with
different study systems and animals, we have some things in common.
First, we had to write a research proposal describing a statistical
approach that we did not yet understand because our adviser suggested it
to us. Second, we have a series of questions we hope to answer to aid in
the management of our systems. And third, we possess (or will collect)
data sets from our systems which we will use to answer our research
questions. For example, I studied population and movement ecology of
game birds and focused on questions relating to the composition and
distribution of wild turkey populations, such as how they use different
land cover types throughout the year. To that end, I collected data on
turkey locations using GPS and VHF radio-telemetry which, in combination
with remotely sensed data products such as the National Land Cover
Database, represented the data I had available to describe my system.

Regardless of the system we wish to study, information can be broken
down into two categories: what is known and what is unknown. In the case
of our turkey example, the knowns are turkey locations, individual
turkey characteristics, and associated land cover at each location. The
unknowns are the underlying relationships that define how these two data
sets relate to one another. Do turkeys select for a given land cover
type consistently more than others? Is there some effect of weather on
selection? Maybe a turkey's sex or age affects their decision-making
process? What about time of day? While in practice these decisions can
be described using increasingly complex modeling techniques, we can
simplify a decision to a weighted coin flip describing whether a turkey
would select one option over another. For the sake of being on the same
page, lets use this system and data as an example throughout the module
to describe these coin flips. Specifically, we wish to know how much the
presence of snow impacts a turkey's use of forested over other land
cover options.

So now we have our data and the research question we wish to answer, but
we still need to determine how we will approach our analysis (ideally
this is done in the planning stage of your research). Within the field
of ecology, we tend to choose an approaches that fall within one of two
paradigms, bayesian or frequentist statistics. While there are many
similarities and distinctions between the two (which we will summarize
at the end of this module), let's take a 10,000 ft view of how each
approaches known and unknown information.

\hypertarget{a-frequentist-approach}{%
\subsubsection{A Frequentist Approach}\label{a-frequentist-approach}}

A frequentist would look at our turkey resource selection question and
say that there is some fixed parameter, \(\theta\), that describes the
effect size of snow depth on selection for forests. By fixed, we mean
that there is some true value for \(\theta\) that applies to our study
system. We do not know it, but it exists (even if it is 0). We then
treat our collected data, y, as random variables which are drawn from
some larger pool of possible data. The data is random because if we were
to go back out and collect new data, we would almost certainly get a
different data set due to any number of factors such as measurement
error, predator presence, or individual behavioral differences. We can
describe the relationship between our known and unknown information
using a probability statement, \(Pr(y|\theta)\), which reads ``the
probability of collecting our observed data assuming some unknwon value
for \(\theta\).''

This probability statement alone doesn't do much for us because we don't
know what \(\theta\) is. However, it can be rewritten as a
\textbf{likelihood function}, \(L(\theta|y)\), which now makes our
parameter dependent on the data we collected. This reads as ``the
likelihood of a parameter value given the data'' and is equivalent to
our probability statement, \(L(\theta|y) = Pr(y|\theta)\). By re-framing
our known and unknown information in this way, we can ask what value of
\(\theta\) is the most likely to be true given the data we collected. By
reversing the dependencies, we can use our model to ask how likely our
data set is for a given value of of \(\theta\). In terms of our research
question, what relationship between turkey habitat selection and snow
best describes the location information that we collected. We can
compare likelihoods across different values for \(\theta\) to find which
value maximizes the likelihood. The exact steps to this process are more
involved and involve some math that I will not cover here, but for more
information search for \textbf{Maximum Likelihood Estimators (MLE)}.

Estimating the value for \(\theta\) that maximizes the likelihood is
only half of what we need to answer our research question. The other
half is defining our uncertainty in that estimate, which can be
described using \textbf{confidence intervals}. A confidence interval is
just the range of values defining \(\theta\) that fall within some
arbitrary level of certainty we wish to apply to the model. Again, the
exact methods for determining confidence intervals are not relevant.
What is important to understand is that any measure of uncertainty we
provide is evaluated and described in terms of the frequency of
hypothetical replicates of our data (hence the term frequentist). If we
were to go out and re-sample turkey locations, what values of \(\theta\)
would result in producing the same or similar data. It is important to
note here, we are still considering \(\theta\) a fixed value, we are
just hedging our bets as to what that value actually is.

\hypertarget{a-bayesian-approach}{%
\subsubsection{A Bayesian Approach}\label{a-bayesian-approach}}

A Bayesian approach is going to start in a very similar place as a
frequentist one. We have known data (y), a parameter we wish to describe
(\(\theta\)), and a model defining our hypothesized relationship between
the two (\(P(\theta|y)\)). A major difference between Bayesian and
Frequentist is that, where a frequentist treats

\hypertarget{bayes-theorem}{%
\paragraph{Bayes Theorem:}\label{bayes-theorem}}

\[ Pr(\theta|y) = \frac{Pr(y|\theta) Pr(\theta)}{Pr(y)}\] {[}What are
theta and y in this scenario{]} {[}Theta = unobserved{]} {[}y =
observed{]}

\hypertarget{in-plain-english}{%
\subsubsection{In plain english\ldots{}}\label{in-plain-english}}

Without focusing on the individual components of the equation, it
states\ldots{}

\hypertarget{the-components}{%
\subsubsection{The Components}\label{the-components}}

{[}Statistical Rethinking pf 36?{]}

Now that we understand what the equation says, lets examine the
individual components and how they help reach that conclusion\ldots{}

\textbf{Prior Distribution} -

\[ \sum_j{Pr(y|\theta_j)} Pr(\theta_j) \]

\hypertarget{why-does-this-work}{%
\subsubsection{Why does this work?}\label{why-does-this-work}}

Bayes theorem centers around the probability of theta and y both
occuring at the same time, otherwise written as \(Pr(y,\theta)\).
{[}Green et al.~has a good walkthrough on pg3{]}

\hypertarget{an-example}{%
\subsubsection{An Example:}\label{an-example}}

To provide a more tangible example, lets apply Bayes theorem to a simple
example. Assume\ldots{}

\hypertarget{bayesian-vs-frequentist}{%
\subsubsection{Bayesian vs Frequentist}\label{bayesian-vs-frequentist}}

So now that you have at least a broad understanding of the differences
in philosophy between the two approaches, lets identify the aspects of
each that are similar and different from one another.

{[}Kery and Schaub - Bayesian pop analysis ch 2.4{]}

{[}Held and Bove - Likelihood and Bayesian Inference ch 3{]}

{[}Hobbs and Hooten - 4.2 Likelihood profiles{]}

{[}Randomness (frequentist) vs uncertainty (Bayesian) - Nothing is
actually random, if we had enough information and a powerful enough
computer, you could predict almost any natural system in theory{]}

\hypertarget{sources-and-further-reading}{%
\subsubsection{Sources and Further
Reading}\label{sources-and-further-reading}}

This module is a synthesis of the sources listed below.\\
   ???

For a more in depth discussion of the topics covered in this module, I
have found the below resources very useful in furhtering my
understanding of statistics.\\

\end{document}
